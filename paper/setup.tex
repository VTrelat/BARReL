\usepackage{xspace}

\newcommand\barrel{BARReL\xspace}

\makeatletter
\newcommand{\rw}[2]{%
	\ifsimplenotes@disable #2%
	\else \fixme[{\normalfont\st{#1} \textcolor{red}{#2}}]%
	\fi%
}
\makeatother

%%%% TikZ 

\usetikzlibrary{shapes.misc, positioning, chains, calc}

%%%% B operators 

\newcommand\bminop{\ensuremath{\operatorname{min}}}
\newcommand\bmaxop{\ensuremath{\operatorname{max}}}
\newcommand\bmin[1]{\ensuremath{\bminop(#1)}}
\newcommand\bmax[1]{\ensuremath{\bmaxop(#1)}}
\newcommand\bdom[1]{\ensuremath{\operatorname{dom}(#1)}}
\newcommand\bapp[2]{\ensuremath{#1(#2)}}
\newcommand\binterval[2]{\ensuremath{#1 .. #2}}
\newcommand\bfinone[1]{\ensuremath{\operatorname{FIN}_1({#1})}}

\newcommand\pfun{\mathrel{\ooalign{\hfil$\mapstochar\mkern5mu$\hfil\cr$\to$\cr}}}
\newcommand\pinj{\mathrel{\ooalign{\hfil$\mapstochar\mkern5mu$\hfil\cr$\rightarrowtail$\cr}}}
\newcommand\tinj{\rightarrowtail}
\newcommand\psurj{\mathrel{\ooalign{\hfil$\mapstochar\mkern5mu$\hfil\cr$\twoheadrightarrow$\cr}}}
\newcommand\tsurj{\twoheadrightarrow}
\newcommand\ndres{\mathbin{\rlap{\hbox{$-$}}{\lhd}}}

\newcommand\leanpf{\cdot\mkern-5mu\cdot\mkern-5mu\cdot}

\newcommand\metavar[1]{\ensuremath{?{#1}}}

%%%% Listings 

\colorlet{keywordcolor}{blue!80!white}
\colorlet{symbolcolor}{keywordcolor}
\colorlet{sortcolor}{orange!90!black}
\colorlet{tacticcolor}{keywordcolor}
\colorlet{attributecolor}{black}
\colorlet{commentcolor}{gray}
\colorlet{stringcolor}{green!70!black}

\lstset{
	language=Lean,
	% Spaces are not displayed as a special character
	showstringspaces=false,
	% keep spaces
	% keepspaces=true,
	% Size of tabulations
	tabsize=2,
	% Enables ASCII chars 128 to 255
	extendedchars=false,
	% Case sensitivity
	sensitive=true,
	% Automatic breaking of long lines
	% breaklines=true,
	% breakatwhitespace=true,
	% Default style fors listingsred
	% basicstyle=\ttfamily\scriptsize,
	% Position of captions is bottom
	% captionpos=b,
	% Full flexible columns
	% columns=[l]fullflexible,
	% Style for (listings') identifiers
	identifierstyle=\textcolor{black},
	% Note : highlighting of Coq identifiers is done through a new
	% delimiter definition through an lstset at the beginning of the
	% document. Don't know how to do better.
	% Style for declaration keywords
	keywordstyle=[1]{\textcolor{keywordcolor}},
	% Style for sorts
	keywordstyle=[2]{\textcolor{sortcolor}},
	% Style for tactics keywords
	keywordstyle=[3]{\textcolor{tacticcolor}},
	% Style for attributes
	keywordstyle=[4]{\textcolor{attributecolor}},
	% Style for strings
	stringstyle={\textcolor{stringcolor}},
	% Style for comments
	commentstyle={\itshape\textcolor{commentcolor}},
	% --> We use `mdframed` for this instead <--
	% backgroundcolor=\color{listingsbg},
	% rulecolor=\color{listingsbg!70!black},
	% frame=l,
	% % frameround=tttt,
	% % framextopmargin=10pt,
	% % framexbottommargin=10pt,
	% framerule=3pt,
	% framesep=8pt,
	% Automatically gobble based on first line
	% autogobble=true,
}

\lstdefinelanguage{b} {
morekeywords=[1]{MACHINE, REFINEMENT, IMPLEMENTATION, REFINES, SEES, INCLUDES, IMPORTS, EXTENDS, SETS, CONSTANTS, VARIABLES, ASSERTIONS, PROPERTIES, INVARIANT, OPERATIONS, PRE, IF, THEN, END, INITIALISATION, ANY, WHERE},
morekeywords=[2]{NATURAL, NAT, NATURAL1, NAT1, INTEGER, INT, BOOL, BOOL, POW, POW1, FIN, FIN1, card, min, max},
morecomment=[l]{//},
showstringspaces=false,
mathescape=true,
keepspaces=true,
breaklines=true,
numbers=left,
numberstyle=\tiny\textcolor{black},
% basicstyle=\small\ttfamily,
keywordstyle=[1]\small\itshape\textcolor{keywordcolor},
keywordstyle=[2]\small\textcolor{sortcolor},
commentstyle=\textcolor{commentcolor},
identifierstyle={\ttfamily\textcolor{black}},
numbersep=6pt,
% frame=single,
columns=[l]fullflexible,
% framerule=.7pt,
% rulecolor=\color{black},
% columns=[l]fullflexible,
% framerule=0pt,
% backgroundcolor=\color{black!5},
% frame=shadowbox,
% rulesepcolor=\color{black!30},
% xleftmargin=1ex,
% xrightmargin=1em,
literate=
	{∈}{{\ensuremath{\in}}}1
{∉}{{\ensuremath{\notin}}}1
{∧}{{\ensuremath{\land}}}1
{∨}{{\ensuremath{\lor}}}1
{¬}{{\ensuremath{\lnot}}}1
{⇒}{{\ensuremath{\implies}}}1
{⇔}{{\ensuremath{\iff}}}1
{≠}{{\ensuremath{\neq}}}1
{≤}{{\ensuremath{\leq}}}1
{≥}{{\ensuremath{\geq}}}1
{→}{{\ensuremath{\to}}}1
{↔}{{\ensuremath{\leftrightarrow}}}1
{∅}{{\ensuremath{\varnothing}}}1
{∪}{{\ensuremath{\cup}}}1
{∩}{{\ensuremath{\cap}}}1
{⊆}{{\ensuremath{\subseteq}}}1
{⊂}{{\ensuremath{\subset}}}1
{⇒}{{\ensuremath{\Rightarrow}}}1
{∀}{{\ensuremath{\forall}}}1
{∃}{{\ensuremath{\exists}}}1
{↣}{{\ensuremath{\tinj}}}1,
extendedchars=true
}

\newcommand\inlineb[1]{\lstinline[language=b, basicstyle=\normalsize\ttfamily]|#1|}