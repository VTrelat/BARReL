\section{Conclusion}
We presented \barrel, a Lean~4 backend for Atelier~B that turns industrial B artefacts into ordinary Lean goals over a natural set-theoretic encoding of the B language.
Our central design choice is to make B's ubiquitous partiality explicit: partial operators are represented as total Lean functions parameterized by proofs of well-definedness which are reified as standard additional goals.
This yields robust proof scripts that are protected against ill-typed or ill-defined instantiations, while remaining close to the structure of the obligations produced by Atelier~B.

We evaluated the approach on a three-level refinement chain for minimum search, showing generated Lean goals, many of which being discharged automatically by \barrel’s lightweight automation, leaving refinement and invariant-preservation obligations to be proved interactively.
From a trust perspective, \barrel reduces the trusted computing base by implementing the translation pipeline within Lean's metaprogramming framework as a natural, almost homomorphic mapping from B constructs to Lean expressions relying only on Atelier~B for proof obligation generation.
Future work includes reducing redundancy among well-definedness goals, e.g.\@ via subsumption and sharing, extending the covered fragment of B and the accompanying lemma library, and integrating a verified proof obligation generator within Lean itself.