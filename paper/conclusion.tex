\section{Conclusion}
We presented \barrel, a Lean~4 backend for Atelier~B that turns industrial B artifacts into ordinary Lean goals over a natural set-theoretic encoding of the B language.
Our central design choice is to make B's ubiquitous partiality explicit: partial operators are represented as total Lean functions parameterized by proofs of well-definedness which are reified as standard additional goals.
This yields robust proof scripts that are protected against ill-typed or ill-defined instantiations, while remaining close to the structure of the obligations produced by Atelier~B.

We evaluated the approach on a three-level refinement chain for minimum search, showing generated Lean goals, many of which being discharged automatically by \barrel's lightweight automation, leaving refinement and invariant-preservation obligations to be proved interactively.
From a trust perspective, \barrel reduces the trusted computing base by implementing the translation pipeline within Lean's metaprogramming framework as a natural, almost identical mapping from B constructs to Lean expressions relying only on Atelier~B for proof obligation generation.
Future work includes reducing redundancy among well-definedness goals, e.g.\@ via subsumption and sharing, extending the covered fragment of B and the accompanying lemma library, and integrating a verified proof obligation generator within Lean itself.

\paragraph*{Artifact availability}
A Zenodo snapshot of the BARReL project is available at \url{https://doi.org/10.5281/zenodo.<RECORD>}.
The archive contains the complete Lean~4 implementation of BARReL (B surface embedding, POG reader, encoder, discharger macros, and tactic layer), together with the sample B machines and Lean scripts used in the evaluation---including the \texttt{MinSearch} refinement chain and the \texttt{DerivFalse} counterexample.
The artifacts are a standard Lean~4 project and pin their dependencies via \texttt{lean-toolchain} and \texttt{lake-manifest.json}; a typical build is \texttt{lake update} followed by \texttt{lake build}.
To replay the end-to-end workflow that regenerates proof obligations from B sources, an Atelier~B installation providing \texttt{bxml} and \texttt{pog} is required; which then has to be pointed inside \barrel by setting the directory containing \texttt{bin/} and \texttt{include/} via \texttt{set\_option barrel.atelierb "<path>"}.