\section{Related Work}
\label{sec:related}

Several approaches have explored using interactive theorem provers to increase confidence in developments written in the B family of methods by moving proof obligations into a prover with a small, auditable kernel.
An early line of work translates (fragments of) B into PVS, leveraging type synthesis to recover the typing information needed by the target logic and then relying on PVS for interactive discharge of the resulting obligations~\cite{Bodeveix2002BtoPVS}.
In the Isabelle/HOL ecosystem, mechanizations have focused primarily on Event-B.
The Isabelle plugin for Rodin~\cite{Schmalz2012RodinIsabelle} provides a formalization of the Event-B logic, including treatment of partial functions via definitional extensions and supporting automation.
A more user-facing shallow embedding was proposed recently, viewing Event-B as a DSL hosted in Isabelle/HOL~\cite{Wolff2024EB_Isabelle}.
Beyond these, the Rocq proof assistant has also been used in complementary work~\cite{Bodeveix2002BtoPVS} that emphasizes mechanized semantics and meta-theory, rather than providing a drop-in backend integrated with an industrial proof-obligation generator.

\barrel is complementary in scope and emphasis.
Rather than formalizing the logic, it targets the existing Atelier~B workflow.
By consuming POG files or invoking the generator, \barrel turns each obligation into a Lean theorem over Mathlib’s set-theoretic primitives~\cite{mathlib4}, while preserving B notations and focusing on usability for B developers.
\barrel also aims at minimizing the learning effort required to get started and thus provides basic automation to reduce user effort, emphasizing on explicit handling of well-definedness conditions.
