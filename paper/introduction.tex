\section{Introduction}

The B method~\cite{Bmethod} and its associated toolsets, notably Atelier~B \cite{AtelierB}, have been used for decades in industrial developments where strong assurance arguments are required.
B specifications are written as \emph{machines} with sets, constants, variables, invariants and operations; Atelier~B generates \emph{proof obligations} (POs) that guarantee, for example, invariant preservation and refinement correctness, and ships with automated and interactive provers tailored to this logic.
In many certified developments, these tools are trusted as part of the verification chain.

In parallel, interactive theorem provers have seen rapid adoption, driven by their expressive type theories, powerful automation, and growing libraries of reusable mathematics and verification components.
Lean~4~\cite{lean} in particular offers a small trusted kernel, a rich standard library, and a meta-programming framework that makes it attractive both as a programming language and a proof assistant.
However, despite B’s long-standing industrial role, support for using modern interactive theorem provers as backends for Atelier~B remains very limited: some previous work targets Isabelle~\cite{IsabelleHOL} but is either no longer maintained~\cite{Schmalz2012RodinIsabelle} or provides specific support~\cite{Wolff2024EB_Isabelle} for Event-B~\cite{AbrialEventB}.
There were also developments in Rocq~\cite{Coq12}, focusing primarily on formalizing the semantics of B rather than providing an integrated backend~\cite{Bodeveix2002BtoPVS}, and Atelier~B’s own interactive prover is comparatively dated and lacks many modern features.

This paper presents \emph{\barrel} (B Automated tRanslation for Reasoning in Lean), a Lean~4 library that bridges Atelier~B and Lean by turning B proof obligations into Lean theorems.
Given a B machine (or an existing POG file), \barrel invokes Atelier~B to generate POs when needed and translates them into Lean theorems expressed over Mathlib~\cite{mathlib4}'s set-theoretic primitives, preserving the original structures and notations as closely as possible.
The library provides commands allowing a user to turn B developments into a sequence of Lean theorems corresponding to the POs of the developments;\@ users then discharge these goals interactively with standard tactic scripts, and the resulting theorems are added to the environment under names derived from the original POG tags.
The translation pipeline is organised into clearly separated components: a lightweight embedding of B syntax, a POG file reader, an encoding layer mapping B constructs to Lean expressions, and a discharger that generates theorem declarations and connects them to user proofs. This high-level architecture is illustrated in \cref{fig:barrel-archi}.

\begin{figure}[t]
	\caption{High-level picture of \barrel.}
	\label{fig:barrel-archi}

	\centering
	\begin{tikzpicture}[
			file/.style={chamfered rectangle, draw, chamfered rectangle corners=north west},
			entity/.style={rectangle, draw},
			node distance=1.5cm and 1.5cm,
			every node/.style={inner sep=6pt, align=center, minimum height=1.22cm, outer sep=0},
			% start chain = going right,
			arr/.style={->, shorten <=3pt, shorten >=3pt}
		]
		\node[file] (B mch) {.mch};
		\node[file, right=of B mch] (B bxml) {.bxml};
		\node[entity, right=of B bxml] (pos) {Proof\\obligations};
		\node[entity, right=of pos] (thms) {Lean\\theorems};

		\draw[arr] (B mch) -- node[above=-4mm, midway] {\scriptsize Parsing} (B bxml);
		\draw[arr] (B bxml) -- node[above=-6mm, midway] {\scriptsize PO\\[-.6ex]\scriptsize generation} (pos);
		\draw[arr] (pos) -- node[above=-6mm, midway] {\scriptsize PO\\[-.6ex]\scriptsize encoding} (thms);
		\draw[arr, shorten <=7pt, shorten >=7pt] (thms.south) arc[start angle=220, end angle=500, x radius=9.5mm, y radius=9.5mm] node[midway, fill=white, minimum height=0pt, inner sep=2pt] {\scriptsize Proving};
	\end{tikzpicture}
\end{figure}

Our contributions are therefore threefold:
\begin{itemize}
	\item \barrel, an open-source lightweight and modular backend for B developments written in Lean~4;
	\item a translation pipeline preserving B notations and proof obligations, properly handling partial operators via explicit well-definedness conditions;
	\item an evaluation on a small refinement chain demonstrating an Atelier~B-style refinement workflow inside Lean, with most well-definedness side-conditions discharged automatically.
\end{itemize}
Finally, \barrel opens the door to using Lean as an independent backend for the B Method and suggests interesting extensions, such as an integrated proof obligation generator---which can even be verified---and even an embedded DSL for writing B developments directly in Lean.
% We do not discuss semantics or soundness of the encoding in this paper.

The rest of this paper is organised as follows.
We first recall the necessary background on the B method, proof obligations, and Lean.
We then present the design and implementation of \barrel, detailing the encoding of B syntax, the interface with Atelier~B's proof obligation generator, and our treatment of partial operators and their well-definedness obligations.
We also discuss basic automation features to reduce user effort in discharging common proof obligations.
This is followed by a case study illustrating the end-to-end workflow of using \barrel on a representative B development.
We conclude with a discussion of future extensions, and position our work with respect to existing approaches to connecting B with modern interactive theorem provers.