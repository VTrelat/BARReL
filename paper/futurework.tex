\section{Future Work and Conclusion}

\noindent\todo[Increase coverage of B]

\paragraph*{Reducing the number of WD side goals via subsumption}
As evidenced by \cref{tab:case-study-stats}, \barrel generates a lot more WD side goals compared to Atelier B.
This stems from a combination of multiple factors: \begin{itemize}
	\item Atelier B is able to reason directly on the machine itself, while \barrel only knows about the obligations generated.
	      Thus, Atelier B can insert WD conditions only where needed (at the call sites of partial operators), once and for all, and share them between sub-goals (or rather not duplicate them).
	\item When invariants are $n$-ary conjunctions, Atelier B's \texttt{pog} binary generates multiple separate obligations (one per conjunct) while duplicating the hypotheses.
	      Since \barrel encodes each obligation individually and separately, this means that WD conditions that come from hypotheses of the obligation (e.g.\@ in invariant preservation theorems) are needlessly duplicated.
\end{itemize}
In practice, although \barrel generates a lot more WD side goals, most are automatically solved by an internal tactic\todo[Just mention again what's been decribed in \cref{par:automation}] made specifically for discharging WD conditions.
The remaining WD side goals, those that are not automatically proved, are usually either totally unrelated or logical consequences of each other.
\todo[blah blah blah implement subsumption and voilà!]

\noindent\todo[More clever automation, even for regular POs]

\noindent\todo[eDSL?]